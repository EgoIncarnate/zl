\chapter{Implementation Status and Performance}
\label{status}

The current ZL prototype supports most of C and an
important subset of C++.  For C, the only major feature not supported
is bitfields, mainly because the need has not arisen.  C++ is a rather
complicated language, and fully implementing it correctly is beyond
the scope of our research.  We aim to implement enough of C++ to
demonstrate our approach; in particular, we support single
inheritance, but currently do not support multiple inheritance,
exceptions, or templates.

As ZL is at present only a prototype compiler, the overall compile
time (in version 0.3) when compared to compiling with GCC 4.4
is 2 to 3 times slower.  However, ZL is designed to have little to no
impact on the resulting code.  ZL's macro system imposes no run-time
overhead.

The ZL compiler transforms higher level ZL into a low-level
S-expression-like language that can best be described as C with
Scheme syntax.  Syntactically, the output is very similar to fully
expanded ZL as shown in Figure \ref{fig}.
The transformed code is then passed to a modified version
of GCC 4.4. When pure C is passed in we are very careful to avoid any
transformations that might affect performance.  The class macro
currently implements a C++ ABI that is comparable to a traditional
ABI, and hence should have no impact on performance.

\section{C Support}

To demonstrate that ZL can support C programs, two well-known programs
were compiled with ZL: bzip2 and gzip.  Bzip2 was compiled without
modifications, but gzip required some minor modification because it was
an older C program and used some C syntax that is not a subset of C++:
K\&R-style function declarations were transformed into the newer ANSI C
style, and one instance of |new| as a variable was renamed to |new_|.

Overall, compile times were 2 to 3 times slower with ZL in comparison
to compiling with GCC 4.4.  However, both programs compiled correctly,
produced correct results, and had similar run times to the
GCC-compiled versions.

\section{C++ Support}
\label{c++-support}

To evaluate ZL's suitability to compile C++ programs, we chose to
compile randprog~\cite{randprog}, which is a small C++ program that
generates random C programs.  Randprog uses inheritance and other
important C++ features, such as overloading and nondefault
constructors.  In addition, it uses a few C++ features that ZL does not
yet support, so we changed randprog in small ways to compensate.  These
changes include reworking the command-line argument parsing, which
used of a library that requires many modern C++ features; explicit
instantiation of |vector| instances; changing uses of the |for_each|
template function into normal |for| loops; and reworking some functions
to avoid returning complex objects.

Randprog was verified to produce correct results by fixing the seed
and comparing the generated program with a version of randprog
compiled with GCC for several different seeds.  It was also
instrumented with Valgrind and found free of memory errors.

Overall compile time was around 2.5 times slower with ZL when compared to
GCC 4.4.  A direct run-time performance comparison is of limited
usefulness, since ZL does not use the same C++ library as GCC, but the
runtime performance of the ZL-compiled version of randprog was up to
twice as fast as the GCC-compiled version.

\section{Debugging Support}
\label{debugging-support}

In error messages, ZL provides a full backtrace of what was expanded
from where in the case of a compile time error involving macros.

ZL provides very basic source level debugging support.  ZL makes an
effort to keep track of line numbers and passes this information onto
the debugger to provide for meaningful backtrace.  
% FIXME: Explain how this is done in the face of complicated macro 
% transformations.
In addition, variable names are also available, but in most cases the
name has been mangled.  For most cases this means 
adding a \verb/$/ and a number to the name.
