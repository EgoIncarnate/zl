\chapter{ZL Tutorial}
\label{zl-tutorial}

This chapter will give you an overview of ZL with a focus on ZL
extensibility features.

\section{Hello World}

ZL basic syntax is the same as it is in C or C++ therefor a simple
Hello World program is nothing but:

\begin{code}
int main() {
  printf("Hello World!")
}
\end{code}

To compile, save this file to |hello.zl| and then use zlc:
\begin{code}
zlc hello.zl
\end{code}
which should create an excitable |a.out| which you can run.

Note that there is no |#include| line in the source file.  The ZL
prelude includes some of the more common functions from the C standard
library and ZL source files are not run through the C preprocessor
by default.

The driver script |zlc| is meant to act as a drop drop in replacement
for GCC.  For example,
\begin{code}
zlc main.zl file1.cpp file2.zlp -o main
\end{code}
compiles the pure ZL source file |main.zl|, the C++ source file
|file1.cpp|, and the ZL source file |file2.zlp| into an executable
|main|.  Each file is compiled slightly differently based on the
extension as ZL has different modes for C, C++, and ZL source files.
In addition C, C++, and ZL source files with the |.zlp| extension are
run through the C preprocessor to handle includes and other
token-level macros that the higher-level ZL macro processor can not
handle.  As already mentioned, pure ZL files with |.zl| extension are
not preprocessed.

\section{Macro Introduction: A Perl Like Or}
\label{pattern-macros}

C's or operator returns a boolean value.  Sometimes it would be more
convenient if it returns the first true value like Perl does.  For
example if we wanted to use the first non-NULL pointer we could use:
\begin{code}
some_call(or(p1,p2))
\end{code}
where |p1| and |p2| are pointers.  Defining |or| as a function (or even
a function template) will not work because the |or| macro should not
evaluate the second argument if the first one evaluates to a true
value.

Fortunately, ZL makes it very easy to define simple macros such as this
using the |macro| form:
\label{or}
\begin{code}
macro or(x, y) { ({typeof(x) t = x; t ? t : y;}); }
\end{code}
(In ZL, as in GCC, the |({...})| is a statement expression whose value
is the result of the last expression, and |typeof(x)| gets the type of
a variable.)  Like Scheme macros~\cite{syn-abst}, ZL macros
are hygienic, which means that they respect lexical scope.  For
example, the |t| used in |or(0.0, t)| and the |t| introduced by the
|or| macro remain separate, even though they have the same symbol
name.

\paragraph{Syntax Macros.}

It is even possible to locally redefine the behavior of the built-in
\verb/||/ operator so that we could instead just use \verb/p1 || p2/.
All that is required to archive this transformation is to change the
|macro| to |smacro|:
\begin{code}
smacro or(x, y) { ({typeof(x) t = x; t ? t : y;}); }
\end{code}
The |smacro| form defines a \textit{syntax macro} which operates on
syntax, as oppose to the |macro| form which defines a \textit{function-call
  macro} in which calls to it take the shape of a function call.

The syntax version of the |or| macro works because ZL performs most of
its operations using interment S-expression like language.  For
example \verb/p1 || p2/ is parsed as |(or p1 p2)|, before being
converted into an AST (details of which are given in Chapter
\ref{parsing})

\paragraph{Keyword Parameters.}

The |or| macro above has too \textit{positional} parameters.  Macros
can also have \textit{keyword} parameters and \textit{default values}.
For example:
\begin{code}
macro sort(list, :compar = strcmp) {...}
\end{code}
defines the macro |sort|, which takes the keyword argument |compar|,
with a default value of |strcmp|.  A call to sort will look something
like {\tt|sort(list,| |:compar = mycmp)|}.

\section{Classes and Templates}
\label{template-intro}

ZL also supports a limited subset of C++ which includes classes and
very basic template support.  Classes are defined if the usual way.  ZL
recognizes the template syntax, but templates are defined using a
macro and must be explicitly instantiated by calling the macro.  For
example, to use a |vector| of type |int|:

\begin{code}
mk_vector(int)

int main() {
  vector<int> con;
}
\end{code}

We will revisit the |mk_vector| macro in Section \ref{mk_vector}.

\section{Extending The Grammar: Foreach}
\label{foreach}

One very common operation is to iterate through the elements of a
container.  It is such a common operation that support for it was
incorporated latest version of the C++ standard via special version of
the |for| loop.  ZL doesn't support this version, but we can very
easily define a similar construct.  Instead of extending the syntax of
|for| we will define a new syntax form: |foreach|.  The usage of the
form is straightforward; for example, the following code prints all the
elements of a vector:
\begin{code}
int main() {
  vector<int> con;
  ...
  foreach (elm in con) {
    printf("%d\n");
  }
}
\end{code}

The first thing we need to do to support the new |foreach| form is to
define some new syntax.  ZL provides basic support for new syntax by
extending the PEG using the |new_syntax| form.  We add support the new
loop form by rewriting the |CUSTOM_STMT| production as follows:
\begin{code}
new_syntax {
  CUSTOM_STMT := _cur / <foreach> "foreach" "(" {ID} "in" {EXP} ")" {STMT};
}
\end{code}
The special |_cur| production refers to the previous version of the
|CUSTOM_STMT| and anything between |{}| becomes a subpart of the
syntax object that is named between the |<>|. 
With the new syntax defined the definition of the macro is fairly
straightforward:

\begin{code}
smacro foreach (VAR, WHAT, BODY) {
  typeof(WHAT) & what = WHAT;
  typeof(what.begin()) i=what.begin(), e=what.end();
  for (; i != e; ++i) {
    typeof(*i) & VAR = *i;
    BODY;
  }
}
\end{code}

Chapter \ref{parsing} gives more details on the PEG and how to
extend it.

\section{Procedural Macros}
\label{foreach-proc}

All the macros shows so far are simple \textit{pattern-based} macros.
Sometimes we need to do more than simply rearrange syntax.  For
example, in the |foreach| macro of the previous section, if the
container lacks a |begin| or |end| method ZL will return a cryptic
error message which will refer to the expanded output, rather than the
user input; it would be far better to give a straightforward error
message which refers to the user input.  For these more complex tasks
ZL provides support for procedural macros which can take action base on
their input.

A procedural macro is a function which maps one syntax object to
another.  ZL provides special syntax support for defending these
functions.  They look a lot like pattern-based macros expect that the
keyword |proc| proceeds the macro and the body of the macro is ZL code
rather than the expanded output.  Syntax is created using special
quasiquote syntax.  For example, the following macro defines a
version of the |foreach| which returns an error message if the
container lacks the proper methods:
\begin{code}
proc smacro foreach (VAR, WHAT, BODY) {
  Syntax * what = ``WHAT;
  if (!symbol_exists(``begin, what) || !symbol_exists(``end, what))
    return error(what, "Container lacks begin or end method.");
  return `{
    typeof(WHAT) & what = WHAT;
    typeof(what.begin()) i=what.begin(), e=what.end();
    for (;i != e; ++i) {typeof(*i) & VAR = *i; BODY;}
  };
}
\end{code}

The |symbol_exists| and |error| are API functions which are callbacks
into the compiler.  The \verb/``/ and \verb/`{/ are quasiquotes; the
first forms quotes an identifier; the second form quotes arbitrary
syntax.  Pattern variables can only be used inside quasiquotes;
however, the |symbol_exists| and |error| function requires a syntax
object.  Hence, \verb/``WHAT/ is used to extract the syntax object
associated with the pattern variable and store it in a local variable.

\paragraph{Antiquotes.}

Rather than using pattern variables, it is possible to use antiquotes
to store the result of the match in local variables by prefixing the
variable with a \verb/$/.  For example, here is a version of foreach
that uses antiquotes instead:
\begin{code}
proc smacro foreach ($var, $what, $body) {
  if (!symbol_exists(``begin, what) || !symbol_exists(``end, what))
    return error(what, "Container lacks begin or end method.");
  return `{
    typeof($what) & what = $what;
    typeof(what.begin()) i=what.begin(), e=what.end();
    for (;i != e; ++i) {typeof(*i) & $var = *i; $body;}
  };
}
\end{code}

When an antiquote is used when matching syntax the results of the
match are stored in a local variable hence \verb/$body/ gets stored in
the local variable |body|, which is then used for by the
|symbol_exist| and |error| function.  When an antiquote is used inside
quasiquotes than it is replaced with the value of the expression; in
most cases this is a local variable.

\paragraph{Additional Info.}

Shown here is only a small taste of what procedural macros can do.  ZL
procedural macros are very powerful and in many ways, act more as
an extension to the compiler than a simple macro that transforms text
from one form to another.  Chapter \ref{proc-macros} gives additional
information on procedural macros, which includes a lower level and
more powerful API.

\section{Templates and Bending Hygiene}
\label{mk_vector}

As already mentioned in Section \ref{template-intro}, ZL does not have builtin
template support.  Instead a macro can be used to create instances of
a template.  Figure \ref{vector-impl} shows the macro and supporting
code to create for the |vector| template class.
\begin{figure}
\begin{codef}
template vector;

macro mk_vector (T,@_) :(*)
{
  __once vector<T>
  class vector<T> {
  private:
    T * data_;
    T * end_;
    T * storage_end_;
  public:
    typedef T * iterator;
    typedef const T * const_iterator;
    typedef size_t size_type;

    iterator begin() const {return data_;}
    iterator end()   const {return end_;}
    bool empty() const {return data_ == end_;}
    size_t size() const {return end_ - data_;}
    size_t max_size() const {return (size_t)-1;}
    size_t capacity() const {return storage_end_ - data_;}
    T & front() const {return *data_;}
    T & back() const {return *(end_-1);}
    T & operator[](size_t n) const {return data_[n];}
    vector<T>() : data_(NULL), end_(NULL), storage_end_(NULL) {}
    vector<T>(size_t n) : ... {resize(n);}
    vector<T>(const vector<T> & other) : ... {...}
    void push_back(const T & el) {...}
    void pop_back() {...}
    void assign(const T * d, unsigned sz) {...}
    void resize(size_t sz) {...}
    void erase(T * pos) {...}
    ~vector<T>() {...}
    ...
  };
}
\end{codef}
\caption{ZL implementation of vector template class}
\label{vector-impl}
\end{figure}

There are two things to note about this macro, the first is some
special syntax to make the template instance to behave as expected,
and the second is to bend normal hygiene rules so that |vector| and the
members of the class are visible outside the macro.

\paragraph{Special Template Syntax.}

Template classes have special syntax that ZL needs to know to
recognize; which is done using the |template| form (which is used
differently than it is in C++).  In addition templates have special
linkage rules so that there are not multiple instances of the same
template in the executable, hence |__once| is used to declare that
duplicate symbols should be merged when linking.

\paragraph{Bending Hygiene.}
\label{macro-export-intro}

Normal hygiene rules will normally prevent the |mk_vector| macro from
working as expected as all symbols created will only be visible to the
macro itself.  To fix this ZL macros supports special syntax,
the~|:(*)|, which will \textit{export} top-level symbols from the
macro (which includes the members of the vector class).  Details of
exactly what is exported and how are given in Section
\ref{macro-export}.

It is also possible to export specific symbols by specifying the symbol
name instead of~|*|, again see Section \ref{macro-export} for the details.

\section{Bending Hygiene Part 2: Fancy Loops}
\label{fluid-intro}

In the previous section we needed to make the macro introduced symbols
visible outside the context of the macro.  Sometimes it is necessary
to make symbols visible to the parameters of the macros.  For this
case, ZL provides a different mechanism, the \textit{fluid binding}.
For example, if we wanted to create a version of the for loop which
also supports a Perl style redo we might try
\begin{code}
macro redo() {goto redo;}

smacro for (INIT, TEST, INC, BODY) {
  for (INIT; TEST; INC) {
    redo():
    BODY;
  }
}
\end{code}
(where the |for| used in the macro will refer to the builtin |for| and
not the macro itself and |redo:| is a label for the goto in the |redo|
macro).  A simple usage of the macro could be:

\begin{code}
int main() {
  for (int i = 0; i < 10; ++i) {
    ++i;
    printf("%d\n", i);
    if (i % 3 == 0)
      redo();
  }
}
\end{code}

Unfortunately this will not work.  The problem is the |redo| label
introduced in the |for| macro is private to that macro.  Exporting the
label will not help, as the label used in the |redo| macro is still
lexically scoped.  The export syntax only has meaning to new binding
forms not existing ones, hence that form can't be used.  ZL does provide a
way to make both |redo| labels totally unhygienic (see Section
\ref{replace_context}), but that solution will not compose well \cite{syn-parm}.

What is really needed is something akin to dynamic scoping in the
hygiene system.  That is, for the |redo| label to be scoped based on
where it is used when expanded, rather than where it is written in the
macro definition.  This can be done by marking the |redo| symbol as
|fluid| using |fluid_label| at the top level and then using |fluid|
when defining the symbol in local scope.  For example, the following
will allow the above to compile:

\begin{code}
fluid_label redo;

macro redo() {goto redo;}

smacro for (INIT, TEST, INC, BODY) {
  for (INIT; TEST; INC) {
    fluid redo:
    BODY;
  }
}
\end{code}

The general form to declare fluid binding is |fluid_binding|, the
details of which are described in Section \ref{fluid}

\section{User Types}
\label{user-types-intro}

ZL does not have a built-in notion of classes, rather the type system
has what is known as a \textit{user type} of which classes are built
from using macros.  A user type has two parts: a type, generally a
|struct|, to hold the data for the class instance, and a collection of
symbols for manipulating the data.

The collection of symbols is a \textit{module}.  For example,
\begin{code}
module M { int x;
           int foo(); }
\end{code}
defines a module with two symbols.  Module symbols are used by either
importing them into the current namespace, or by using the
special syntax |M::x|, which accesses the |x| variable in the above
module.

A user type is created by using the |user_type| primitive, which
serves as the module associated with the user type.  A type for the
instance data is specified using |associate_type|.

As an example, the class\footnote{For simplicity, we leave off access
  control declarations and assume all members are public in this
  work when the distinction is unimportant.}
\begin{code}
class C { int i; 
          int f(int j) {return i + j;} };
\end{code}
roughly expands to:
\begin{code}
user_type C {
  struct Data {int i;};
  associate_type struct Data;
  macro i (:this ths = this) {...}
  macro f(j, :this ths = this) {f`internal(this, j);}
  int f`internal(...) {...}
}
\end{code}
which creates a user type |C| to represent a class |C|; the structural
type |Data| is used for the underlying storage.  The macro |i|
implements the |i| field, while the |f| macro implements the |f|
method by calling the function |f`internal| with |ths| as the first
parameter.  Additional details of how classes are implemented are
given in Chapter \ref{class-user-type}.

