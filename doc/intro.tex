\chapter{Introduction}
\label{intro}

C is a simple language that gives the user considerable control
over how source code maps to machine code.  For example, structures are
guaranteed to be laid out in a particular way, dynamic memory is not
allocated unless it is asked for, all function calls are explicit, and
the only functions created are the ones that are explicitly defined.
For this reason, most low-level system code is written in C.  Nevertheless
programmers want to use high-level language constructs for various
reasons, and such use generally relinquishes low-level control.
For example, C++ does not guarantee a particular layout of objects.
This lack of control can cause a number of problems, including
compatibility problems between different releases of software and
between different compilers.

A programmer can regain control over higher-level language feature
implementation through an extensible compiler.  To provide such
extensibility, macros for C are an ideal choice.  Macros for C let a
programmer build higher-level constructs from a small core, rather
than forcing a programmer to accept a built-in
implementation. Moreover, since macros elevate language extensions to
the level of a library, individual advanced language features are only
active when loaded.

A simple macro system, such as the C preprocessor, is not adequate for
this purpose, nor is any macro system that acts simply as a
preprocessor.  Rather, the macro system must be an integral part of
the language that can do more than rearrange syntax.  In addition, the
C base language must be extended to support the necessary primitives
for implementing higher-level features.  ZL, a new C-compatible and
C++-like systems programming language in development, addresses both
of these needs.

This document presents ZL.  Chapter \ref{zl-tutorial} gives an into of
ZL features in the form of a tutorial.  The other chapters give a
detail overview of ZL features.  The appendices give information on
some of ZL's implementation.

The document is a work in progress.  Proofreading of this document is
greatly appreciated.  The easiest thing is to just edit the source
(please, no reformatting!) and submit a pull request with
your corrections.

%% ZL Goals:

%% \begin{itemize}
%% \item Customizable
%% \item Control
%% \end{itemize}
