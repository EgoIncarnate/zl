\chapter{Roadmap}
\label{roadmap}

The last chapter gave an idea of where ZL is now, this chapter will
give an idea of where I see ZL heading.

\section*{High Level Goals}

\begin{itemize}

\item

In addition to supporting C and C++ syntax, support ZL specific
syntax.  The ZL specific syntax will be like C but provide more sane
syntax for declaration for example:
\begin{code}
var x : int = 20;
fun f (x : int, y : int) : int {...}
\end{code}
Also when in ZL mode all code will be put into a module by default so
that functions (and likely types) don't need to be declared before
they are used.

\item

Extend User Types be able to support algebraic data type, which
includes support for pattern matching |case|s.  There is a lot of
overlap between classes and algebraic data types and I intend to
unify the two concepts.

\item

Once algebraic data types are implemented, work on rewriting the ZL
compiler in ZL itself.  The ZL compiler uses a lot of advanced C++
features that won't be implemented it ZL yet; however, it is my hope
that with the combination of advanced macros and algebraic data types
I will be able to find alternative---and maybe even better---means of
archiving the same goal.

\end{itemize}

